\documentclass[11pt,a4paper]{article}
\usepackage[margin=42pt,landscape]{geometry}

\usepackage{multicol}
\setlength{\columnsep}{31pt}
\setlength{\columnseprule}{0.4pt}

\usepackage{titlesec}
\usepackage{titling}
\usepackage{fontspec}
% Specify different font for section headings
\newfontfamily\headingfont[]{Gill Sans}
\titleformat*{\section}{\LARGE\headingfont}
\titleformat*{\subsection}{\Large\headingfont}
\titleformat*{\subsubsection}{\large\headingfont}
\renewcommand{\maketitlehooka}{\headingfont}

\usepackage{lipsum}

\begin{document}
\begin{multicols}{3}
\title{Haskell Cheat Sheet}
\author{
  Matthijs Ooms\\
  \texttt{mo@tty.nl}
  \and
  Michel Rijnders\\
  \texttt{mies@tty.nl}
}
\date{
  Joy of Coding\\
  March 1, 2013
}
\maketitle

\section*{Data Types}

\subsection*{Characters and Strings}

\begin{itemize}
\item \verb!'a'! - single character
\item \verb!"abc"! - Unicode string
\end{itemize}

\subsection*{Numbers}

\begin{itemize}
\item \verb!42! - integer
\item \verb!3.14! - floating point
\end{itemize}

\subsection*{Booleans}

\begin{itemize}
\item \verb!True!
\item \verb!False!
\end{itemize}

\subsection*{Lists}

\begin{itemize}
\item \verb![]! - empty list
\item \verb![1,2,3]! - list of three integers
\item \verb!1 : 2 : 3 : []! - alternate way to write a list using
  ``cons'' (\verb!:!)
\item \verb!"abc"! - list of three characters (strings are lists)
\end{itemize}

\subsection*{Tuples}

\begin{itemize}
\item \verb!(1, "a")! - 2 element tuple of a number and a string
\item \verb!(1, 'a', 'b', 'c')! - 4 element tuple of a number and 3
  characters
\end{itemize}

\section*{Operators}

\begin{itemize}
\item \verb!+! - addition
\item \verb!-! - subtraction
\item \verb!/! - division
\item \verb!*! - multiplication
\item \verb!==! - equals
\item \verb!<! - less than
\item \verb!<=! - less than, or equals
\item \verb!>! - greater than
\item \verb!>=! - greater than, or equals
\item \verb!++! - list concatenation
\end{itemize}

\section*{Functions}

Functions are defined by declaring their name, and arguments, and a
equals sign:
\begin{verbatim}
square x = x * x
\end{verbatim}

Function names must start with a lowercase letter or an underscore.

\section*{Modules}

A module is a compilation unit which exports functions. To make a
Haskell file (\verb!.hs!) a module, add a module declaration to the
top:
\begin{verbatim}
module OurModule where
\end{verbatim}

Module names must start with a uppercase letter.

\subsection*{Exports}

If an export list is not provided, then all functions are
exported. Limiting what is exported is achieved by adding a list of
names before the \verb!where! keyword:

\begin{verbatim}
module OurModule (f,g,...) where
\end{verbatim}

\subsection*{Imports}

To import evertyhing exported by a module just use the module name:

\begin{verbatim}
import Data.List
\end{verbatim}

Importing selectively is achieved by giving a list of names:

\begin{verbatim}
import Data.List (intersperse)
\end{verbatim}

\section*{GHCi}

\begin{itemize}
\item \verb!:cd <dir>! - change directory
\item \verb!:load <module>! - load module
\item \verb!:reload! - reload all modules
\item \verb!:help! - show help
\item \verb!:quit! - exit
\end{itemize}

\section*{Pattern Matching}

Multiple ``clauses'' of a function can be defined by
``pattern-matching'' on the values of arguments.

\begin{verbatim}
agree "y" = "Great!"
agree "n" = "Too bad."
agree _ = "Huh?"
\end{verbatim}

\subsection*{Lists}

\begin{itemize}
\item \verb!(x:xs) = [1,2,3]! binds \verb!x! to \verb!1!
\item \verb!(x:_) = [1,2,3]! binds \verb!x! to \verb!1! and \verb!xs! to  \verb![2,3]!
\item \verb!(_:xs) = [1,2,3]! binds \verb!xs! to \verb![2,3]!
\item \verb!(x:y:z:[]) = [1,2,3]! binds \verb!x! to \verb!1!, \verb!y! to \verb!2!, and \verb!z! to \verb!3!
\end{itemize}

Note that the empty list only matches the empty list.

\section*{Currying}

Functions do not have to get all their arguments at once. Consider the
following function:

\begin{verbatim}
add x y = x + y
\end{verbatim}

Using add we can now write functions that add a certain value:
\begin{verbatim}
addTwo = add 2
five = addTwo 3
\end{verbatim}

\section*{Anonymous Functions}

An anonymous function (i.e. a lambda expression or lambda for short),
is a function without a name. They can be defined at any time like so:

\begin{verbatim}
addTwo = \x -> x + 2
\end{verbatim}

\section*{Local Functions}

Local functions can be defined within a function using \verb!let!. The
\verb!let! keyword must always be followed by \verb!in!.

\begin{verbatim}
five = let addTwo = add 2
       in addTwo 3
\end{verbatim}

Local functions can also be defined using \verb!where!:

\begin{verbatim}
five = addTwo 3
       where addTwo = add 2
\end{verbatim}

\section*{List Comprehensions}

A list comprehension creates a list of values based on the generators
and guards given:

\begin{verbatim}
f xs = [x * x | x <- xs, mod x 2 == 0]
\end{verbatim}

Another example:

\begin{verbatim}
up cs = [c | c <- cs, isUpper c]
\end{verbatim}

\section*{Case}

\verb!case! is similar to a \verb!switch! statement in C or Java, but
can match a pattern.

\begin{verbatim}
agree x =
  case x of
    "y" -> "Great!"
    "n" -> "Too bad."
    _ -> "Huh?"
\end{verbatim}

\section*{If, Then, Else}

As opposed to languages like C and Java \verb!if, then, else! is an
expression, not a control statement, i.e. it always returns a value:

\begin{verbatim}
agree x =
  if x == "y"
    then "Great!"
    else "Too bad."
\end{verbatim}

\end{multicols}
\end{document}
